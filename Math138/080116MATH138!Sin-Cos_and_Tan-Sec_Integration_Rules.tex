\documentclass{letter}
\usepackage[margin=0.75in]{geometry}
\usepackage{amsmath}
\usepackage{amssymb}
\usepackage{enumerate}
\usepackage{changepage}
\usepackage{tikz}
\usepackage{pgfplots}
\pgfplotsset{compat=1.8}

\pgfplotsset{vasymptote/.style={
		before end axis/.append code={
			\draw[densely dashed] ({rel axis cs:0,0} -| {axis cs:#1,0})
			-- ({rel axis cs:0,1} -| {axis cs:#1,0});
		}
	}}
	
\newcommand{\m}{\begin{bmatrix}}
\newcommand{\mm}{\end{bmatrix}}
\newcommand{\0}[1]{\begin{bmatrix}#1\end{bmatrix}}
\newcommand{\h}[1]{\underline{\textbf{#1}}}	

\begin{document}
	\begin{center}
		\LARGE Math138 - January 08, 2016\\
		\large Sin/Cos and Tan/Sec Integration Rules
	\end{center}
	\vspace{0.25 in}
	
	\h{Rules}
	
	In the last lecture, we went over the 4 cases for sin/cos integration. We refer to these cases when solving the following examples.
	
	P.s. If the letters not being on the same line as the math bothers you, sorry! I know there are ways to fix it but I've been typesetting for hours and I don't care anymore :D
	
	\h{Examples}
	\begin{enumerate}[a)]
		\item \begin{minipage}[t]{0.3\textwidth}
			\begin{flalign*}
				&\int \sin^6 x \cos^3 x\; dx&\\
				&= \int u^6 \cos^2 x\; du\\
				&= \int u^6(1-u^2)\; du\\
				&= \int u^6 - u^8\; du\\
				&= \frac{u^7}{7} - \frac{u^9}{9} + c\\
				&\text{Side note: Can they take marks off if you write $-c$?}			
			\end{flalign*}
		\end{minipage}
		\begin{minipage}[t]{0.5\textwidth}
			\begin{flalign*}
				u &= \sin x&\\
				du &= \cos x\; dx\\
			\end{flalign*}
		\end{minipage}
		\item \begin{minipage}[t]{0.3\textwidth}
			\begin{flalign*}
			&\int \cos ^15 x \sin ^5 x\; dx&\\
			&= -\int u^15 \sin ^4 x \; du\\
			&= -\int u^15(1-u^2)^2\;du\\
			&= -\int u^15(1-2u^2 + u^4)	\; du\\
			&= -\int u^16 - 2u^16 + u^19\\
			&= -(\frac{u^16}{16} - \frac{u^18}{9} + \frac{u^20}{20}) + c\\
			&= \frac{\cos^16 x}{16} - \frac{\cos^18 x}{9} + \frac{\cos^20 x}{20} + c
			\end{flalign*}
		\end{minipage}
		\begin{minipage}[t]{0.5\textwidth}
			\begin{flalign*}
			u &= \cos x&\\
			du &= -\sin x\; dx\\
			\sin^4 x& = (1-\cos^2 x)^2\\
			(\sin^2 x)^2 &= (1-u^2)^2\\
			\end{flalign*}
		\end{minipage}
		\clearpage
		\item \begin{minipage}[t]{0.3\textwidth}
			\begin{flalign*}
			&\int \sin^2 x \cos^2 x\; dx&\\
			&= \int\left( \frac{1-\cos(2x)}{2}\right)\left( \frac{1+\cos(2x)}{2} \right)\\
			&= \frac{1}{4} \int 1-\cos^2(2x)\; dx\\
			&= \frac14 \int \sin^2 (2x)\; dx\\
			&= \frac14 \int \left( \frac{1 - \cos(4x)}{2}\right)\; dx\\
			&= \frac18 \int 1 - \cos 4x\; dx\\
			&= \frac18 \left( x - \frac{\sin(4x)}{4} \right) + c
			\end{flalign*}
		\end{minipage}
		\begin{minipage}[t]{0.5\textwidth}
			\begin{flalign*}
			u &= \cos x&\\
			du &= -\sin x\; dx\\
			\sin^4 x& = (1-\cos^2 x)^2\\
			(\sin^2 x)^2 &= (1-u^2)^2\\
			\end{flalign*}
		\end{minipage}
	\end{enumerate}
	
	\h{Powers of Sec/Tan (Same as Csc/Cot)}
	
	Say we are trying to integrate $\int \tan^m x \cdot \sec^n x$. The rules work very similarly to sin/cos, but the cases are different.
	
	\begin{enumerate}[\text{Case } 1:]
		\item If $n$ is even and $m$ is anything, let $u = \tan x$ and use the identity $\sec^2 x = 1 + \tan^2 x$
		\item If $m$ is odd and $n$ is anything, let $u = \sec x$ and use the identity $\tan^2 x = \sec^2 x - 1$\\
		Note: If $n$ is even and $m$ is odd, use the higher power for $u$.
		\item if $m$ is even and $n$ is odd, use $\tan^2 x = \sec^2 x - 1$ to write the entire integral in terms of $\sec$, then use the formula for $\int \sec^n x\; dx$ from Assignment 1.
	\end{enumerate}
	
	\h{Examples}
	
	\begin{enumerate}[a)]
		\item \begin{minipage}[t]{0.3\textwidth}
			\begin{flalign*}
			&\int \tan^2 x \sec^4 x\; dx&\\
			&= \int u^2 \sec^2 x\; dx\\
			&= \int u^2(1+\tan^2 x) \; du\\
			&= \int u^2(1+u^2)\; du\\
			&= \int u^2 + u^4\; du\\
			&= \frac{u^3}{3} + \frac{u^5}{5}\\
			&= \frac{\tan^3 x}{3} + \frac{\tan^5 x}{5}
			\end{flalign*}
		\end{minipage}
		\begin{minipage}[t]{0.5\textwidth}
			\begin{flalign*}
			u &= \tan x&\\
			du &= \sec^2 x\; dx\\
			\end{flalign*}
		\end{minipage}
		\item \begin{minipage}[t]{0.3\textwidth}
			\begin{flalign*}
				&\int \tan^3 x \sec^3 x\; dx&\\
				&= \int u^2 \tan^2 x \; du\\
				&= u^2(\sec^2 x - 1)\; du\\
				&= u^2(u^2 - 1)\; du\\
				&= \int u^4 - u^2\; du\\
				&= \frac{u^5}{5} - \frac{u^3}{3} + c\\
				&= \frac{\sec^5 x}{5} - \frac{\sec{^3 x}}{3}
			\end{flalign*}
		\end{minipage}
		\begin{minipage}[t]{0.5\textwidth}
			\begin{flalign*}
			u &= \sec x&\\
			du &= \sec x \tan x; dx\\
			\end{flalign*}
		\end{minipage}
		\item \begin{minipage}[t]{0.3\textwidth}
			\begin{flalign*}
			&\int \tan^3 x \sec^{14} x\; dx&\\
			&= \int u^{13} \tan^2 x \; du\\
			&= \int u^{13}(\sec^2 x - 1)\; du\\
			&= \int u^{13}(u^2 - 1)\; du\\
			&= \int u^{26} - u^{13}\; du\\
			&= \frac{u^{27}}{27} - \frac{u^{14}}{14} + c\\
			&= \frac{\sec^{27} x}{27} - \frac{\sec^{14} x}{14} + c
			\end{flalign*}
		\end{minipage}
		\begin{minipage}[t]{0.5\textwidth}
			\begin{flalign*}
			u &= \sec x&\\
			du &= \sec x \tan x\; dx\\
			\end{flalign*}
		\end{minipage}
		\item \begin{minipage}[t]{0.3\textwidth}
			\begin{flalign*}
			&\int \tan^4 x \sec^3 x\; dx&\\
			&= \int(\sec^2 x - 1)^2 \sec^3 x\; dx\\
			&= \int(\sec^4 + 2\sec^2 x + 1) \sec^3 x\; dx\\
			&= \int sec^6 x + 2\sec^5 x + \sec^3 x \; dx\\
			&\text{Now use the reduction formula for secant to reduce (From Assignment 1)}
			\end{flalign*}
		\end{minipage}
		\begin{minipage}[t]{0.5\textwidth}
			
		\end{minipage}
	\end{enumerate}
\end{document}