\documentclass{letter}
\usepackage[margin=0.75in]{geometry}
\usepackage{amsmath}
\usepackage{amssymb}
\usepackage{enumerate}
\usepackage{changepage}
\usepackage{tikz}
\usepackage{pgfplots}
\pgfplotsset{compat=1.8}

\pgfplotsset{vasymptote/.style={
		before end axis/.append code={
			\draw[densely dashed] ({rel axis cs:0,0} -| {axis cs:#1,0})
			-- ({rel axis cs:0,1} -| {axis cs:#1,0});
		}
	}}

\begin{document}
	\begin{center}
		\LARGE Math138 - January 04, 2016\\
		\large Integration by Parts
	\end{center}
	\vspace{0.25 in}
	
	\underline{\textbf{The Formula}}
	
	Let $u$ and $v$ be functions of $x$.
	
	The product rule says $\frac{d}{dx} (u)(v) = \frac{du}{dx} v + u \frac{dv}{dx}$.
	
	If we integrate, we get:
	\begin{flalign*}
		\int \frac{d}{dx}(u\;v) dx &= \int \frac{du}{dx} \cdot (v)(dx) + \int u \cdot \frac{du}{dx} dx&\\
		(u)(v) &= \int v \cdot (du) + \int u \cdot (dv)\\
		&= \int u \cdot dv = (u)(v) - \int v \cdot du\;\;\;\; \text{(REMEMBER THIS FORMULA IT'S VERY IMPORTANT!)}
	\end{flalign*}
	
	This looks a little bit confusing, and I was confused at first so I'll try to explain. What we've just proven is this: If we have a function that is very difficult or impossible to integrate with our current rules, we can split it into the product of two functions $u$ and $dv$ and use \underline{Integration by Parts} to integrate.
	
	To do this, first we need a $u$ and $dv$ function! Even though $dv$ looks like a derivative, don't worry about this. Choose $u$ and $dv$ such that $u$ times $dv$ equals the function you're trying to integrate. Below are tips for choosing effective functions. Then you'll want to \textit{differentiate} $u$ and call the derivative $du$. Next, \textit{integrate} $dv$ and call the integral $v$. Now by Integration by Parts, the original integration we were trying to solve is equal to $(u)(v) - \int v \cdot du$.
	
	\underline{\textbf{Strategies for Choosing Functions}}
	
	When integrating the product of two functions using Integration by Parts, we need to choose $u$ and $v$ effectively. We will be differentiating $u$ and integrating $v$.
	\begin{itemize}
		\item Pick $dv$ to be the most complicated part of the integral that you know how to integrate.
		\item Pick $u$ so it gets simpler when you take the derivative.
	\end{itemize}
	
	\underline{\textbf{ILATE Rule}}
	
	The above tips may seem ambiguous and confusing. Thankfully, the 'ILATE' rule make choosing $u$ and $dv$ as simple as memorization.
	
	\begin{itemize}
		\item Pick $u$ to be the first function that appears in this list:\begin{itemize}
			\item \textbf{I}nverse trig functions
			\item \textbf{L}ogarithms
			\item \textbf{A}lgebraic (Polynomials)
			\item \textbf{T}rig
			\item \textbf{E}xponentials
		\end{itemize}
	\end{itemize}
	
	\clearpage
	
	\underline{\textbf{Examples}}
	
	\begin{enumerate}[a)]
		\item \begin{minipage}[t]{0.3\textwidth}
			\begin{flalign*}
				&\int x^2 \ln x \; dx&\\
				&= \frac{x^3}{3} \ln x - \int \frac{x^3}{3} \cdot \frac{1}{x} dx\\
				&= \frac{x^3}{3} \ln x - \int \frac{x^2}{3} dx\\
				&= \frac{x^3}{3} \ln x - \frac{x^3}{9} + c
			\end{flalign*}
		\end{minipage}
		\begin{minipage}[t]{0.5\textwidth}
			\begin{flalign*}
				u &= \ln x&\\
				du &= \frac{1}{x} dx\\\\
				dv &= x^2 dx\\
				v &= \frac{x^3}{3}
			\end{flalign*}
		\end{minipage}
		
		\item \begin{minipage}[t]{0.3\textwidth}
			\begin{flalign*}
				&\int x e^x \; dx&\\
				&= x e^x - \int e^x \; dx\\
				&= x e^x - e^x + c
			\end{flalign*}
		\end{minipage}
		\begin{minipage}[t]{0.5\textwidth}
			\begin{flalign*}
				u &= x&\\
				du &= \frac{1}{x} dx\\\\
				dv &= x^2 dx\\
				v &= \frac{x^3}{3}
			\end{flalign*}
		\end{minipage}
		
		\item \begin{minipage}[t]{0.3\textwidth}
			\begin{flalign*}
				&\int \ln x \; dx&\\
				&= x \ln x - \int x \frac{1}{x} \; dx\\
				&= x \ln x - x + c
			\end{flalign*}
		\end{minipage}
		\begin{minipage}[t]{0.5\textwidth}
			\begin{flalign*}
				u &= \ln x&\\
				du &= \frac{1}{x} dx\\
				dv &= dx\\
				v &= x
			\end{flalign*}
		\end{minipage}
		
		\item \begin{minipage}[t]{0.4\textwidth}
			\begin{flalign*}
				&\int x^2 \cos x\; dx&\\
				&=x^2 \sin x - \int 2x \sin x \; dx\\
				&= x^2 \sin x - ((2x)(-\cos x) - 2\int \cos x \; dx\\
				&= x^2 \sin x + 2x \cos x + 2 \sin x + c
			\end{flalign*}
		\end{minipage}
		\begin{minipage}[t]{0.5\textwidth}
			\begin{flalign*}
				u &= x^2,\;\; du = 2x\;dx \;\;\;\text{FIRST USAGE}&\\
				dv &= \cos x,\;\; v = \sin x\\
				u &= 2x,\;\; du = 2 \; dx \;\;\;\text{SECOND USAGE}\\
				dv &= 2\; dx,\;\;v = -\cos x
			\end{flalign*}
		\end{minipage}
		
		\item \begin{minipage}[t]{0.3\textwidth}
			\begin{flalign*}
			&\int e^x \cos x \; dx\\
			&= \cos x e^x - \int e^x (-\sin x) \; dx\\
			&= e^x \cos x  + \int e^x \sin x \; dx\\
			&= e^x \cos x + e^x \sin x - \int e^x \cos x\\
			&\text{Notice the integral is the same as our original integral.}\\
			&\text{Let the original integral be $I$}\\
			I &= e^x \cos x  + e^x \sin x - I\\
			2I &= e^x \cos x + e^x \sin x\\
			I &= \frac{1}{2}(e^x \cos x + e^x \sin x)
			\end{flalign*}
		\end{minipage}
		\begin{minipage}[t]{0.5\textwidth}
			\begin{flalign*}
			u &= \cos x, \;\; du = -\sin x\; dx \;\;\;\text{FIRST USAGE}&\\
			dv &= e^x, \;\; v = e^x\\
			u &= \sin x,\;\; du = \cos x\;\;\; \text{SECOND USAGE}\\
			dv &= e^x,\;\; v = e^x 
			\end{flalign*}
		\end{minipage}
		
		\item \begin{minipage}[t]{0.3\textwidth}
			\begin{flalign*}
			&\int_1^3 x^3 \ln x\; dx\\
			&= \left[\frac{x^4}{4} \ln x \right]_1^3 - \int_1^3 \frac{x^4}{4}\frac{1}{x}\; dx\\
			&= \left[\frac{x^4}{4} \ln x \right]_1^3 - \int_1^3 \frac{x^3}{4} \; dx\\
			&= \left[ \frac{x^4}{4} \ln x - \frac{x^4}{16}\right]_1^3\\
			&= \frac{81}{4} \ln 3 - \frac{81}{16} - (0 - \frac{1}{16})\\
			&= 18 \frac{\ln 3}{4} - \frac{80}{16}
			\end{flalign*}
		\end{minipage}
		\begin{minipage}[t]{0.5\textwidth}
			\begin{flalign*}
			u &= \ln x&\\
			du &= \frac{1}{x} dx\\\\
			dv &= x^3 dx\\
			v &= \frac{x^4}{4}
			\end{flalign*}
		\end{minipage}
	\end{enumerate}
	
	\underline{\textbf{Trig Identities}}
	
	In this section, we have two goals: integrating powers of Sin/Cos, and Sec/Tan. When these are multiplied together, they don't reduce into simpler trig functions like most other combinations.
	
	\underline{\textbf{How to Integrate Powers of Sin/Cos}}
	
	e.g. $\int \sin^m (x) \cos^n (x)\;dx$
	
	\begin{enumerate}[\text{Case }1:]
		\item $m$ and $n$ are both even.\\
		\begin{itemize}
			\item Use trig formulas to reduce to lower powers.
			\item Then use these trig identites:
			\begin{itemize}
				\item $\sin^2 x = \frac{1}{2}(1-\cos(2x))$
				\item$\cos^2 x = \frac{1}{2}(1 + \cos(2x))$
			\end{itemize}
		\end{itemize}
		
		\item $m$ or $n$ is odd.\\\\
		Then substitute $u = $ the base of the even power.
		
		\item Both are odd.\\\\
		Let $u = $ the base of the higher power.
		
		\item $m$ or $n$ isn't an integer.\\\\
		Let $u$ be the base of the ugliest power.
	\end{enumerate}
\end{document}