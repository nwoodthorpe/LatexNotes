\documentclass{letter}
\usepackage[margin=0.75in]{geometry}
\usepackage{amsmath}
\usepackage{amssymb}
\usepackage{enumerate}
\usepackage{changepage}

\begin{document}
	\begin{center}
		\LARGE Math137 - November 227, 2015\\
		\large Change of Variable - More Examples
	\end{center}
	\vspace{0.25 in}
	
	\begin{itemize}
		\item[Ex. ] Let $F(x) = \int_{2x}^{x^3} e^{-t} dt$. Find $F'(2)$.\\\\
		This looks very similar to examples we did in the last note with FTC1. We have an issue though. To use FTC1 to find $F'(x)$, we need to have a constant lower bound and a function upperbound in terms of $x$. Here, we have two bounds both functions of $x$. Using integral property 4, we can introduce some arbitrary constant and rewrite our integral in terms of it as follows:
		\begin{flalign*}
			F(x) &= \int_{2x}^0 e^{-t} dt + \int_0^{x^3} e^{-t} dt&\\
			&= - \int{0}^{2x} e^{-t} dt + \int_0^{x^3} e^{-t} dt&\\
			F'(x) &= -(e^{2x}) \cdot \frac{d}{dx} 2x + e^{-x^{3}} \cdot \frac{d}{dx} x^3\\
			&= -2e^{-2x} + ex^2 e^{-x^3}\\
			F'(2) &= -2e^{-2(2)} + e(2)^2 e^{-8}\\
			&= -2e^{-4} + 12e^{-8}
		\end{flalign*}
	\end{itemize}
	
	\underline{\textbf{Using a Change of Variable to Evaluate Anti-Derivatives}}\\
	
	\begin{itemize}
		\item[Ex. ] Evaluate:\\
		\begin{enumerate}[a)]
			\item $\int_0^1 (\sqrt{x^2 + x + 3})(2x+1) dx$\\\\\
			\begin{minipage}[t]{0.5\textwidth}
				Let $u = x^2 + x + 3$\\
				$\frac{du}{dx} = 2x + 1$\\
				$du = (2x+1)dx$\\
				BOUNDS:\\
				When $x=0, u = x^2 + 0 + 3 = 3$\\
				When $x=1, u = 1^2 + 1 + 3 = 5$
				\begin{flalign*}
					&\int_0^1 (\sqrt{x^2 + x + 3})(2x+1) dx&\\
					&= \int_3^5 \sqrt{u}\;\; du\\
					&= \left[ \dfrac{u^{\frac{3}{2}}}{\frac{3}{2}}\right]_3^5\\
					&= \left[\dfrac{2}{3} \sqrt{u^3} \right]_3^5\\
					&= \dfrac{2}{3}(\sqrt{125} - \sqrt{27})
				\end{flalign*}
			\end{minipage}
			\begin{minipage}[t]{0.5\textwidth}
				\textbf{SIDE WORK/THOUGHT PROCESS}\\
				Notice we don't know the function that gave this derivative. We cannot expand because of the square root, and we don't have a product rule for integration\\
				We will try a new method: Change of Variable. We will let some expression $= u$ such that the expression become simple enough for us to anti-differentiate.\\
				Figuring out the correct value is tough, and requires practice.\\
				Let $u = x^2 + x + 3$\\
				$\frac{du}{dx} = 2x + 1$\\
				$du = (2x+1) dx$\\
				This is what we want. This substitution will give us a $du$ which we need, and will eliminate all other terms. Lovely.\\
				We will also need to change our bounds of integration, since we are working with $u$ and not $x$ anymore. Just sub the x value into our equation for $u$ to get the new bounds.
			\end{minipage}
			\clearpage
			\item $\int \dfrac{x}{\sqrt{9x^2 + 4}} dx$\\\\
			Gross. Again, we don't know a function that differentiates to this, so lets try a variable substitution.\\\\
			Let $u = 9x^2 + 4$\\
			$\dfrac{du}{dx} = 18x$\\
			$du = 18x \cdot dx$\\
			Notice how substituting in $u$ will replace the denominator of our fraction, but leave $x$ and $dx$. We should manipulate the derivative of $u$ in a way that it will eliminate those.\\
			$du = 18x \cdot dx$\\
			$x \cdot dx = \frac{1}{18} \cdot du$\\
			We are ready to attempt the substitution.
			\begin{flalign*}
				&\int \dfrac{x}{\sqrt{9x^2 + 4}} dx &\\
				&= \int \dfrac{1}{\sqrt{u}} \cdot \frac{1}{18} \cdot du\\
				&= \frac{1}{18} \int u^{\frac{-1}{2}} \cdot du\\
				&= \frac{1}{18} \cdot \dfrac{u^{\frac{1}{2}}}{\frac{1}{2}} + C\\
				&= \frac19 \sqrt u + C\\
				&= \frac{1}{9} \sqrt{9x^2 + 4} + C
			\end{flalign*}
			\item $\int 5x e^{-x^2} dx$\\\\
			Let $u = x^2$\\
			$du = 2x \cdot dx$\\
			$\frac52 du = 5x \cdot dx$
			\begin{flalign*}
				&\int e^{-u} \cdot (\frac52 du)&\\
				&= \frac52 \int e^{-u}\\
				&= \frac52 \dfrac{e^{-u}}{-1} + C\\
				&= \frac{-5}{2} e^{-x^2} + C
			\end{flalign*}
			
			\item $\int \dfrac{e^x}{1+e^{2x}} dx$\\\\
			Let $u = e^x$\\
			$\frac{du}{dx} = e^x$\\
			$du = e^x \cdot dx$
			\begin{flalign*}
				&\int \dfrac{e^x}{1+e^{2x}} dx&\\
				&= \int \dfrac{1}{1+u^2} du\\
				&= \arctan(u) + C\\
				&= \arctan(e^x) + C
			\end{flalign*}
		\end{enumerate}
	\end{itemize}
\end{document}