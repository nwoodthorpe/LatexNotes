\documentclass{letter}
\usepackage[margin=0.75in]{geometry}
\usepackage{amsmath}
\usepackage{amssymb}
\usepackage{enumerate}

\begin{document}
	\begin{center}
		\LARGE Math137 - November 2'nd, 2015
	\end{center}
	\vspace{0.25 in}
	\underline{\textbf{NOTE:}}
	
	Today, we had a different prof teach our class. She spoke quieter than her chalk hitting the board, so I picked up next to nothing. Because of this, this note is a \textbf{TAD} short.
	
	\vspace{0.25 in}
	\underline{\textbf{Recap: Extreme Value Theorem}}

	If $f$ is continuous on a closed interval $[a, b]$, then $f$ attains an absolute max $(c, f(c))$ and an absolute minimum $(d, f(d))$ at values $c, d$ in the interval $[a, b]$.
	
	\vspace{0.25 in}
	\underline{\textbf{Extreme Value Theorem: Constant Functions}}
	
	If $f(x) = c$, where $c$ is some constant, all points on the function in any closed interval are simultaneously absolute maximums and absolute minimums.
	
	\vspace{0.25 in}
	\underline{\textbf{Fermat's Local Extreme Theorem}}
	
	If $f$ has a local extreme max or extreme min at the point $c$ and $f'(c)$ exists, then $f'(c) = 0$.
	
	
	
	
	
\end{document}