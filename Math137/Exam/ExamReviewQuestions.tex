\documentclass{letter}
\usepackage[margin=0.75in]{geometry}
\usepackage{amsmath}
\usepackage{amssymb}
\usepackage{enumerate}
\usepackage{changepage}

\begin{document}
	\begin{center}
		\LARGE Math137 - Exam Review - Fall 2015\\
		\large Question List
	\end{center}
	\vspace{0.25 in}
	
	\large\underline{\textbf{Assignment 1:}}
	\begin{itemize}
		\item[] \textbf{Appendix A}\\
		
		Rewrite the following expressions without the absolute value bars.\\
		\begin{minipage}[t]{0.3\textwidth}
			5. $|\sqrt5 - 5|$
		\end{minipage}
		\begin{minipage}[t]{0.3\textwidth}
			9. $|x+1|$
		\end{minipage}
		\begin{minipage}[t]{0.3\textwidth}
			12. $|1-x^2|$
		\end{minipage}\\
		
		Solve the inequality in terms of intervals.\\
		\begin{minipage}[t]{0.3\textwidth}
			18. $1+5x > 5-3x$
		\end{minipage}
		\begin{minipage}[t]{0.3\textwidth}
			26. $(2x+3)(x-1) \geq 0$
		\end{minipage}
		\begin{minipage}[t]{0.3\textwidth}
			37. $\dfrac{1}{x} < 4$
		\end{minipage}\\
		
		Solve the equation for x.\\
		45. $|x+3| = |2x+1|$\\
		
		Solve the inequality.\\
		\begin{minipage}[t]{0.3\textwidth}
			51. $|x+5| \geq 2$
		\end{minipage}
		\begin{minipage}[t]{0.3\textwidth}
			55. $1 \leq |x| \leq  4$
		\end{minipage}
		\begin{minipage}[t]{0.3\textwidth}
			56. $0 < |x-5| < \frac12$
		\end{minipage}\\
		
		\textbf{Appendix B}\\
		
		3. Find the equation of a circle that is centered at the origin and passes through (4, 7).\\
		
		Identify the type of curve and give a rough sketch.\\
		\begin{minipage}[t]{0.3\textwidth}
			12. $y^2 - x^2 = 1$\\
			18. $y = x^2 + 1$\\
			25. $9(x-1)^2 + 4(y-2)^2 = 36$
		\end{minipage}
		\begin{minipage}[t]{0.3\textwidth}
			14. $x = -2y^2$\\
			23. $xy = 4$\\
		\end{minipage}
		\begin{minipage}[t]{0.3\textwidth}
			16. $25x^2 + 4y^2 = 100$\\
			24. $y = x^2 + 2x$
		\end{minipage}\\
		
		\textbf{1.1}\\
		
		Find the domain.\\
		\begin{minipage}[t]{0.3\textwidth}
			33. $f(t) = \sqrt[3]{2t-1}$
		\end{minipage}
		\begin{minipage}[t]{0.3\textwidth}
			37. $f(p) = \sqrt{2-\sqrt{p}}$
		\end{minipage}\\
		
		\textbf{1.3}\\
		
		31. Find a) $f+g$ \;\;\;\;b) $f-g$\;\;\;\; c) $fg$ \;\;\;\;d) $\frac{f}{g}$ \;\;\;\;and state domains.\\
		$f(x) + x^3 + 2x^2$\;\;\;\;\;\; $g(x) = 3x^2 - 1$\\
		
		Express the function in the form $f \circ g$\\
		\begin{minipage}[t]{0.3\textwidth}
			45. $\dfrac{\sqrt[3]{x}}{1+\sqrt[3]{x}}$
		\end{minipage}
		\begin{minipage}[t]{0.3\textwidth}
			47. $v(t) = \sec(t^2) \tan(t^2)$
		\end{minipage}\\
		
		\textbf{1.4}\\
		
		Find the domain of each function.
		
		19. a) $f(x) = \dfrac{1 - e^{x^2}}{1-e^{1-x^2}}$ \;\;\;\;\;\;\; b) $f(x) = \dfrac{1+x}{e^{\cos x}}$
	\end{itemize}
	\clearpage
	\large\underline{\textbf{Assignment 2:}}
	\begin{itemize}
		\item[] \textbf{1.5}\\
		
		Determine if the following function is one-to-one.\\
		\begin{minipage}[t]{0.6\textwidth}
			10. $f(x) = x^4 - 16$\\
			14. $f(t)$ is your age in years.\\
		\end{minipage}
		\begin{minipage}[t]{0.3\textwidth}
			13. $f(t)$ is the height of a football after being kicked.\\
		\end{minipage}
		
		Find the formula for the inverse of the function.\\
		\begin{minipage}[t]{0.3\textwidth}
			21. $f(x) = 1 + \sqrt{2+3x}$
		\end{minipage}
		\begin{minipage}[t]{0.3\textwidth}
			23. $f(x) = e^{2x-1}$
		\end{minipage}
		\begin{minipage}[t]{0.3\textwidth}
			26. $y = \dfrac{1-e^{-x}}{1+e^{-x}}$\\
		\end{minipage}
		
		Find the exact value of each expression.\\
		\begin{minipage}[t]{0.3\textwidth}
			35. a) $\log_2 32$\\
			36. a) $\log_5 \frac{1}{125}$
		\end{minipage}
		\begin{minipage}[t]{0.3\textwidth}
			b) $log_8 2$\\
			b) $\ln(\frac{1}{e^2})$\\
		\end{minipage}
		
		Solve each equation for $x$.\\
		\begin{minipage}[t]{0.3\textwidth}
			51. a) $e^{7-4x} = 6$
		\end{minipage}
		\begin{minipage}[t]{0.3\textwidth}
			53 a) $2^{x-5}$
		\end{minipage}
		\begin{minipage}[t]{0.3\textwidth}
			b) $\ln x + \ln(x-1) = 1$\\
		\end{minipage}\\
		
		Solve each inequality for $x$.\\
		\begin{minipage}[t]{0.3\textwidth}
			55. b) $e^x > 5$
		\end{minipage}
		\begin{minipage}[t]{0.3\textwidth}
			56. b) $1 - 2\ln x < 3$\\
		\end{minipage}
		
		Find the exact value of each expression.\\
		\begin{minipage}[t]{0.3\textwidth}
			64. a) $\tan^{-1} \sqrt3$\\
			67. b) $\sec^{-1} 2$
		\end{minipage}
		\begin{minipage}[t]{0.3\textwidth}
			b) $\arctan(-1)$\\
			68. a) $\arcsin (\sin \frac{5\pi}{4})$
		\end{minipage}
		\begin{minipage}[t]{0.3\textwidth}
			67. a) $\cot^{-1}(-\sqrt3)$\\
			b) $\cos(2\sin^{-1}(\frac{5}{13}))$\\
		\end{minipage}\\
		
		Simplify.\\
		\begin{minipage}[t]{0.3\textwidth}
			70. $\tan(\sin^{-1} x)$
		\end{minipage}
		\begin{minipage}[t]{0.3\textwidth}
			71. $\sin(\tan^{-1} x)$
		\end{minipage}
		\begin{minipage}[t]{0.3\textwidth}
			72. $\sin (2 \arccos x)$
		\end{minipage}\\
		
		\textbf{3.11}\\
		
		Evaluate.\\
		\begin{minipage}[t]{0.2\textwidth}
			1. a) $\sinh 0$
		\end{minipage}
		\begin{minipage}[t]{0.2\textwidth}
			b) $\cosh 0$
		\end{minipage}
		\begin{minipage}[t]{0.2\textwidth}
			4. a) $\sinh 4$
		\end{minipage}
		\begin{minipage}[t]{0.2\textwidth}
			b) $\sinh(\ln 4)$\\
		\end{minipage}\\
		
		Prove the identity.\\
		\begin{minipage}[t]{0.5\textwidth}
			9. $\cosh x + \sinh x = e^x$
		\end{minipage}
		\begin{minipage}[t]{0.5\textwidth}
			13. $\coth^2 x - 1 = \text{csch}^2 x$\\
		\end{minipage}\\
		
		\textbf{Appendix D}\\
		
		Convert from degrees to radians.\\
		\begin{minipage}[t]{0.5\textwidth}
			1. $210\deg$
		\end{minipage}
		\begin{minipage}[t]{0.5\textwidth}
			5. $900\deg$\\
		\end{minipage}\\
		
		Convert from radians to degrees.\\
		\begin{minipage}[t]{0.5\textwidth}
			4. $4 \pi$
		\end{minipage}
		\begin{minipage}[t]{0.5\textwidth}
			8. $\frac{-7 \pi}{2}$\\
		\end{minipage}\\
		
		Prove the identity.\\
		\begin{minipage}[t]{0.5\textwidth}
			47. $\sec y - \cos y = \tan y \sin y$\\
			53. $\sin x \sin(2x) + \cos x \cos(2x) = \cos x$
		\end{minipage}
		\begin{minipage}[t]{0.5\textwidth}
			50. $2 \csc(2t) = \sec t \csc t$\\
			54. $\sin^2 x - \sin^2 y = \sin(x+y) \sin(x-y)$
		\end{minipage}\\
	\end{itemize}
	\clearpage
	\large\underline{\textbf{Assignment 3:}}
	\begin{itemize}
		\item[] \textbf{2.2}\\
		
		Sketch an example of a function that satisfies all of the given conditions.\\
		15. $\displaystyle \lim_{x \to 0^-} f(x) = -1 \;\;\;\; \lim_{x \to 0^+} f(x) = 2 \;\;\;\; f(0) = 1$\\
		16. $\displaystyle \lim_{x \to 0} f(x) = 1 \;\;\;\; \lim_{x \to 3^-} f(x) = -2 \;\;\;\; \lim_{x \to 3^+} f(x) = 2 \;\;\;\; f(0) = -1 \;\;\;\; f(3) = 1$\\
		
		\textbf{2.3}\\
		
		Evaluate the limit, if it exists (USING THE DEFINITION OF A LIMIT)\\
		\begin{minipage}[t]{0.3\textwidth}
			11. $\displaystyle \lim_{x \to 5} \dfrac{x^2 - 6x + 5}{x-5}$\\
			20. $\displaystyle \lim_{t \to 1} \dfrac{t^4 - 1}{t^3 - 1}$
		\end{minipage}
		\begin{minipage}[t]{0.3\textwidth}
			15. $\displaystyle \lim_{t \to -3} \dfrac{t^2 - 9}{2t^2 + 7t + 3}$\\
			27. $\displaystyle \lim_{x \to 16} \dfrac{4 - \sqrt x}{16x - x^2}$
		\end{minipage}
		\begin{minipage}[t]{0.3\textwidth}
			19. $\displaystyle \dfrac{x+2}{x^3 + 8}$\\
			31. $\displaystyle \dfrac{(x+h)^3 - x^3}{h}$
		\end{minipage}\\
	\end{itemize}
	\clearpage
	\large\underline{\textbf{Assignment 4:}}
	\begin{itemize}
		\item[] \textbf{2.5}\\
		
		Use the definition of continuity and the properties of limits to show the function is continuous at the given number $a$.\\
		\begin{minipage}[t]{0.5\textwidth}
			11. $f(x) = (x+2x^3)^4\;, \; a = -1$\\
			13. $p(v) = 2\sqrt{3v^2 + 1}\;, \; a = 1$
		\end{minipage}
		\begin{minipage}[t]{0.5\textwidth}
			12. $g(t) = \frac{t^2 + 5t}{2t + 1}\;, \; a = 2$\\
			14. $f(x) = 3x^4 - 5x + \sqrt[3]{x^2 = 4}\;, \; a = 2$\\
		\end{minipage}\\
		
		Use the intermediate value theorem to show that there is a root of the given equation in the specified interval.\\
		\begin{minipage}[t]{0.5\textwidth}
			53. $x^4 + x - 3 = 0 \;\;\; (1, 2)$\\
			55. $e^x = 3 - 2x \;\;\; (0, 1)$\\
		\end{minipage}
		\begin{minipage}[t]{0.5\textwidth}
			54. $\ln x = x - \sqrt x \;\;\; (2, 3)$\\
			56. $\sin x = x^2 - x \;\;\; (1, 2)$
		\end{minipage}\\
		
		Prove the equation has at least one real root.\\
		57. $\cos x = x^3$\\
		
		69. Is there a number that is exactly one more than it's cube?\\
		
		\textbf{2.6}\\
		
		Sketch the graph of a function that satisfies the following.\\
		7. $\displaystyle \lim_{x \to 2} f(x) = -\infty \;\;\;\; \lim_{x \to \infty} f(x) = \infty \;\;\;\; \lim_{x \to -\infty} f(x) = 0 \;\;\;\; \lim_{x \to 0^+} f(x) = \infty \;\;\;\; \lim_{x \to 0^-} f(x) = -\infty$\\
		8. $\displaystyle \lim_{x \to \infty} f(x) = 3 \;\;\;\; \lim_{x \to 2^-} f(x) = \infty \;\;\;\; \lim_{x \to 2^+} f(x) = - \infty \;\;\;\; f$ is odd.\\
		
		Find the limit or prove it doesn't exist.\\
		\begin{minipage}[t]{0.3\textwidth}
			20. $\displaystyle \lim_{t \to \infty} \dfrac{ t - t\sqrt{t}}{2t^{\frac{3}{2}} + 3t - 5}$\\
			35. $\displaystyle \lim_{x \to \infty} \arctan e^x$
		\end{minipage}
		\begin{minipage}[t]{0.3\textwidth}
			25. $\displaystyle \lim_{x \to \infty} \dfrac{\sqrt{x + 3x^2}}{4x+1}$\\
			36. $\displaystyle \lim_{x \to \infty} \dfrac{e^x - e^{-x}}{e^{3x} + e^{-3x}}$
		\end{minipage}
		\begin{minipage}[t]{0.3\textwidth}
			26. $\displaystyle \lim_{x \to \infty} \dfrac{x + 3x^2}{4x-1}$
		\end{minipage}\\
	\end{itemize}
	\clearpage
	\large\underline{\textbf{Assignment 5:}}
	\begin{itemize}
		\item[] \textbf{2.7}\\
		
		22. If the tangent line to $y=f(x)$ at $(4, 3)$ passes through the point $(0, 2)$, find $f(4)$ and $f'(4)$.\\
		
		Find $f'$.\\
		
		\begin{minipage}[t]{0.5\textwidth}
			33. $f(t) = \dfrac{2t + 1}{t+3}$
		\end{minipage}
		\begin{minipage}[t]{0.5\textwidth}
			36. $f(x) = \dfrac{4}{\sqrt{1-x}}$\\
		\end{minipage}\\
		
		Each limit represents the derivative of some function $f$ at some number $a$. Find $f$ and $a$.\\
		\begin{minipage}[t]{0.3\textwidth}
			37. $\displaystyle \lim_{h \to 0} \dfrac{\sqrt{9+h} - 3}{h}$
		\end{minipage}
		\begin{minipage}[t]{0.3\textwidth}
			39. $\displaystyle \lim_{x \to 2} \dfrac{x^6 - 64}{x-2}$
		\end{minipage}
		\begin{minipage}[t]{0.3\textwidth}
			42. $\displaystyle \lim_{\theta \to \frac{\pi}{6}} \dfrac{\sin \theta - \frac12}{\theta - \frac{\pi}{6}}$\\
		\end{minipage}\\
		
		Find the derivative using the definition of a derivative.\\
		\begin{minipage}[t]{0.3\textwidth}
			21. $f(x) = 3x - 8$\\
			28. $f(x) = \dfrac{x^2 - 1}{2x - 3}$
		\end{minipage}
		\begin{minipage}[t]{0.3\textwidth}
			25. $f(x) = x^2 - 2x^3$\\
			29. $G(x) = \dfrac{1-2t}{3+t}$\\
		\end{minipage}
		\begin{minipage}[t]{0.3\textwidth}
			27. $g(x) = \sqrt{9-x}$\\
			31. $f(x) = x^4$
		\end{minipage}\\
		
		59. Show that $f(x) = |x-6|$ is not differentiable at $x=6$\\
		
		\textbf{Chapter 2 Review:}\\
		
		3. State the following limit laws:
		\begin{enumerate}[a)]
			\item Sum Law
			\item Difference Law
			\item Constant Multiple Law
			\item Product Law
			\item Quotient Law
			\item Power Law
			\item Root Law\\
		\end{enumerate}
		
		4. What does the squeeze theorem say?\\
		
		9. What does the intermediate value theorem say?\\
		
		15. \begin{enumerate}[a)]
				\item What does it mean for $f$ to be differentiable at $a$?
				\item What is the relation between the differentiability and continuity of a function?
				\item Sketch the graph of a function that is continuous but not differentiable at $x = 2$
			\end{enumerate}
	\end{itemize}
	\clearpage
	\large\underline{\textbf{Assignment 6:}}
	\begin{itemize}
		\item[] \textbf{3.1}\\
		
		Differentiate the function using differentiation rules.\\
		\begin{minipage}[t]{0.3\textwidth}
			11. $g(t) = 2t^{\frac{-3}{4}}$
		\end{minipage}
		\begin{minipage}[t]{0.3\textwidth}
			16. $h(t) = \sqrt[4]{t} - 4e^t$
		\end{minipage}
		\begin{minipage}[t]{0.3\textwidth}
			30. $D(t) = \dfrac{1+16t^2}{(4t)^3}$\\
		\end{minipage}\\
		
		Find the equation of the tangent line and normal line at a given point.\\
		\begin{minipage}[t]{0.5\textwidth}
			37. $y = x^4 + 2e^x \;\;\;\; (0, 2)$
		\end{minipage}
		\begin{minipage}[t]{0.5\textwidth}
			38. $y^2 = x^3 \;\;\;\; (1, 1)$\\
		\end{minipage}\\
		
		\textbf{3.2}\\
		
		Find $f'$ and $f''$\\
		\begin{minipage}[t]{0.5\textwidth}
			27. $f(x) = (x^3 + 1)e^x$
		\end{minipage}
		\begin{minipage}[t]{0.5\textwidth}
			28. $f(x) = \sqrt x e^x$
		\end{minipage}\\
		
		\textbf{3.3}\\
		
		Differentiate.\\
		\begin{minipage}[t]{0.3\textwidth}
			9. $y = \dfrac{x}{2- \tan x}$
		\end{minipage}
		\begin{minipage}[t]{0.3\textwidth}
			10. $y = \sin \theta \cos \theta$
		\end{minipage}
		\begin{minipage}[t]{0.3\textwidth}
			14. $y = \dfrac{\sin t}{1 + \tan t}$\\
		\end{minipage}\\
		
		Find the equation of the tangent line to the curve at a given point.\\
		\begin{minipage}[t]{0.5\textwidth}
			21. $y = \sin x + \cos x \;\;\;\; (0, 1)$
		\end{minipage}
		\begin{minipage}[t]{0.5\textwidth}
			22. $y = e^x \cos x \;\;\;\; (0, 1)$\\
		\end{minipage}\\
		
		Find the limit.\\
		\begin{minipage}[t]{0.3\textwidth}
			39. $\displaystyle \lim_{x \to 0} \dfrac{\sin 5x}{3x}$
		\end{minipage}
		\begin{minipage}[t]{0.3\textwidth}
			45. $\displaystyle \lim_{\theta \to 0} \dfrac{\sin \theta}{\theta + \tan \theta}$
		\end{minipage}
		\begin{minipage}[t]{0.3\textwidth}
			48. $\displaystyle \lim_{x \to 0} \dfrac{\sin^2 x}{x}$\\
		\end{minipage}\\
		
		\textbf{3.4}\\
		
		Differentiate.\\
		\begin{minipage}[t]{0.2\textwidth}
			27. $10^{2\sqrt t}$\\
		\end{minipage}
		\begin{minipage}[t]{0.2\textwidth}
			30. $\tan^2(n \theta)$
		\end{minipage}
		\begin{minipage}[t]{0.2\textwidth}
			37. $\cot^2(\sin \theta)$\\
		\end{minipage}
		\begin{minipage}[t]{0.2\textwidth}
			39. $\tan(\sec(\cos \theta))$
		\end{minipage}\\
		
		\textbf{3.5}\\
		
		Differentiate ($\frac{dy}{dx})$\\
		\begin{minipage}[t]{0.5\textwidth}
			7. $x^4 + x^2y^2 + y^3 = 5$\\
			15. $e^{x/y} = x - y$
		\end{minipage}
		\begin{minipage}[t]{0.5\textwidth}
			12. $\cos (xy) = 1 + \sin y$\\
			20. $\tan(x - y) = \dfrac{y}{1+x^2}$\\
		\end{minipage}\\
		
		\textbf{3.11}\\
		
		Find the numerical value.\\
		\begin{minipage}[t]{0.3\textwidth}
			1. a) $\sinh 0$\\
			4. b) $\sinh (\ln 4)$
		\end{minipage}
		\begin{minipage}[t]{0.3\textwidth}
			1. b) $\cosh 0$\\
			5. a) $\text{sech} 0$
		\end{minipage}
		\begin{minipage}[t]{0.3\textwidth}
			4. a) $\sinh 4$\\
			5. b) $\cosh^{-1} 1$\\
		\end{minipage}\\
		
		Find the derivative.\\
		\begin{minipage}[t]{0.5\textwidth}
			33. $h(x) = \sinh(x^2)$\\
			41. $y = \cosh^{-1} \sqrt x$
		\end{minipage}
		\begin{minipage}[t]{0.5\textwidth}
			37. $y = e^{cosh 3x}$\\
			44. $y = $ sech$^{-1} (e^{-x})$
		\end{minipage}\\
	\end{itemize}
	\clearpage
	\large\underline{\textbf{Assignment 7:}}
	\begin{itemize}
		\item[] \textbf{3.6}\\
		
		Differentiate.\\
		\begin{minipage}[t]{0.3\textwidth}
			5. $f(x) = \ln(\frac{1}{x})$\\
			15. $F(s) = \ln(\ln s)$
		\end{minipage}
		\begin{minipage}[t]{0.3\textwidth}
			8. $f(x) = \log_{10} \sqrt x$\\
			18. $y = \ln(\csc x - \cot x)$\\
		\end{minipage}
		\begin{minipage}[t]{0.3\textwidth}
			10. $g(t) = \sqrt{1+ \ln t}$
		\end{minipage}\\
		
		Use logarithmic differentiation.\\
		\begin{minipage}[t]{0.5\textwidth}
			39. $(x^2 + 2)^2 (x^4+4)^4$\\
			44. $y = x^{cos x}$
		\end{minipage}
		\begin{minipage}[t]{0.5\textwidth}
			43. $y = x^x$\\
			47. $y = (\cos x)^x$\\
		\end{minipage}\\
		
		\textbf{3.9}\\
		
		14. If a snowball melts so that it's surface area decreases at a rate of $1cm^3$/min, find the rate at which the diameter decreases when the diameter is 10 cm.\\
		
		17. Two cars start moving from the same point. One travels south at 60 mi/h, the other travels north at 25 mi/h. At what rate is the distance between the cars changing 2 hours later?\\
		
		18. A spotlight on the ground shines on a wall 12m away. If a man 2m tall walks from the spotlight towards the wall at a speed of 1.6m/s, how fast is the length of his shadow on the building decreasing when he is 4m from the building?\\
		
		\textbf{3.10}\\
		
		Find the linearizion at $a$.\\
		\begin{minipage}[t]{0.5\textwidth}
			2. $f(x) = \sin x \;\;\;\; a = \frac{\pi}{6}$
		\end{minipage}
		\begin{minipage}[t]{0.5\textwidth}
			3. $f(x) = \sqrt x \;\;\;\; a = 4$
		\end{minipage}\\
		
		Use linear approximation to estimate.\\
		\begin{minipage}[t]{0.3\textwidth}
			23. $(1.999)^4$
		\end{minipage}
		\begin{minipage}[t]{0.3\textwidth}
			25. $\sqrt[3]{1001}$
		\end{minipage}
		\begin{minipage}[t]{0.3\textwidth}
			27. $e^{0.1}$\\
		\end{minipage}\\
		
		\textbf{4.1}\\
		
		Find the absolute maximum/minimum values of $f$ on the interval.\\
		\begin{minipage}[t]{0.5\textwidth}
			50. $f(x) = x^3 - 6x^2 + 5 \; \left[ -3, 5 \right]$\\
			61. $f(x) = \ln(x^2 + x - 1) \; \left[ -1, 1\right]$\\
		\end{minipage}
		\begin{minipage}[t]{0.5\textwidth}
			54. $f(x) = \dfrac{x}{x^2 - x + 1} \; \left[0, 3 \right]$
		\end{minipage}
		
		\textbf{4.2}\\
		
		State the Mean Value Theorem.\\
		
		Verify the function satisfies the hypothesis of the MVT on the interval. Then, find all numbers $c$ that satisfy the conclusion the MVT.\\
		
		\begin{minipage}[t]{0.5\textwidth}
			11. $f(x) = 2x^2 - 3x + 1 \;\;\; \left[0, 2\right]$\\
			13. $f(x) = \ln x \;\;\; \left[1, 4 \right]$
		\end{minipage}
		\begin{minipage}[t]{0.5\textwidth}
			12. $f(x) = x^3 - 3x + 2 \;\;\; \left[ -2, 2\right]$\\
			14. $f(x) = \frac{1}{x} \left[ 1, 3\right]$
		\end{minipage}
	\end{itemize}
	\clearpage
	\large\underline{\textbf Assignment 8:}
	\begin{itemize}
		\item[] \textbf{4.3}\\
		
		a) Find the intervals on which $f$ is increasing or decreasing.\\
		b) Find the local maximum and minimum values of $f$.\\
		c) Find the intervals of concavity and inflection points.\\
		11. $f(x) = x^4 - 2x^3 + 3$\\
		14. $f(x) = \cos ^2 x - 2 \sin x \;\;\;\;\; 0 \leq x \leq 2\pi$\\
		15. $f(x) = e^{2x} + e^{-x}$\\
		
		Sketch the graph.\\
		\begin{minipage}[t]{0.3\textwidth}
			37. $f(x) = x^3 - 12x + 2$
		\end{minipage}
		\begin{minipage}[t]{0.3\textwidth}
			43. $F(x) = x \sqrt{6-x}$
		\end{minipage}
		\begin{minipage}[t]{0.3\textwidth}
			45. $C(x) = x^{1/3}(x+4)$\\
		\end{minipage}\\
		
		76. For which values of $a$ and $b$ is $2, 2.5)$ an inflection point of the curve $x^2 y + ax + by = 0$? What additional inflection points does the curve have?\\
		
		\textbf{4.4}\\
		
		Find the limit. Use L'Hospitals rule where appropriate. If there is a more elementary method, consider using it.\\
		\begin{minipage}[t]{0.5\textwidth}
			14. $\displaystyle \lim_{x \to 0} \frac{\tan (3x)}{\sin (2x)}$\\
			24. $\displaystyle \lim_{x \to 0} \frac{8^t - t^t}{t}$\\
			43. $\displaystyle \lim_{x \to \infty}x \sin(\pi / x)$\\
			63. $\displaystyle \lim_{x \to \infty} x^{1/x}$\\
		\end{minipage}
		\begin{minipage}[t]{0.5\textwidth}
			19. $\displaystyle \lim_{x \to \infty}\frac{\ln x}{\sqrt x}$\\
			28. $\displaystyle \lim_{x \to 0} \frac{\sinh x - x}{x^3}$\\
			49. $\displaystyle \lim_{x \to 1^+} \ln x \tan(\pi x/2)$\\
		\end{minipage}
		
		Sketch the curve.\\
		\begin{minipage}[t]{0.3\textwidth}
			33. $y = \sin^3 x$\\
			52. $y = \frac{\ln x}{x^2}$
		\end{minipage}
		\begin{minipage}[t]{0.3\textwidth}
			43. $y = \frac{1}{1+e^{-x}}$\\
			53. $y = e^{\arctan x}$
		\end{minipage}
		\begin{minipage}[t]{0.3\textwidth}
			49. $y = \ln(\sin x)$\\
			54. $y = \tan^{-1}(\frac{x-1}{x+1})$
		\end{minipage}\\
	\end{itemize}
	\clearpage
	\large\underline{\textbf Assignment 9:}
	\begin{itemize}
		\item[] \textbf{4.7}\\
		
		28. Find the area of the largest trapezoid that can be inscribed in a circle of radius 1 whose base has a diameter of the circle.\\
		
		35. The top and bottom margins of a poster are each 6cm and the side margins are 4cm. If the area of the printed material on the poster is fixed at 384 cm$^2$, find the dimensions of the poster with the smallest area.\\
		
		42. A cone-shaped paper drinking cup is to be made to hold 27cm$^3$ of water. Find the height and radius of the cone that will use the smallest amount of paper.\\
		
		48. A boat leaves a dock at 2:00 PM and travels due south at a speed of 20km/h. Another boat has been heading due east at 15km/h and reaches the same dock at 3:00PM. At what time were the two boats closest together?\\
		
		56. At which points on the curve $y=1 + 40x^3 - 3x^5$ does the tangent line have the largest slope?\\
		
		58. What is the smallest possible area of the triangle that is cut off by the first quadrant and who's hypotenuse is tangent the the parabola $y = 4 - x^2$ at some point?\\
		
		\textbf{4.8}\\
		
		35. a) Use Newtons Method to find the critical numbers of the function $f(x) = x^6 - x^4 + 3x^3 - 2x$ correct to one iteration.\\
		35. b) Find the absolute minimum value of $f$ correct to two iterations.
	\end{itemize}
	\clearpage
	\large\underline{\textbf Assignment 10:}
	\begin{itemize}
		\item[] \textbf{4.9}\\
		
		Find the most general anti-derivative of the function.\\
		\begin{minipage}[t]{0.3\textwidth}
			6. $f(x) = (x-5)^2$\\
			11. $f(x) = 3\sqrt x - 2\sqrt[3] x$\\
			19. $f(x) = 2^x + 4 \sinh x$\\
		\end{minipage}
		\begin{minipage}[t]{0.3\textwidth}
			9. $f(x) = \sqrt{3}$\\
			13. $f(x) = \frac15 - \frac2x$\\
		\end{minipage}
		\begin{minipage}[t]{0.3\textwidth}
			10. $f(x) = e^2$\\
			16. $R(x) = \sec x \tan x\; 2e^x$
		\end{minipage}\\
		
		Find $f$.\\
		39. $f''(x) = -2 + 12x - 12x^2, \;\; f(0) = 4,\;\; f'(0) = 12$\\
		31. $f''(x) = \sin x + \cos x, \;\; f(0) = 3, \;\;f'(0) = 4$\\
		47. $f''(x) = x^{-2},\;\;\; x>0,\;\;f(1) = 0,\;\;f(2) = 0$\\\
		48. $f'''(x) = \cos x, \;\; f(0) = 1,\;\; f'(0) = 2,\;\;f''(0) = 3$\\
		
		Find an expression for the area under the graph of $f$ as a limit. Do not evaluate the limit.\\
		21. $f(x) = \frac{2x}{x^2+1},\;\;\;\;1 \leq x \leq 3$\\
		23. $f(x) = \sqrt{\sin x}, \;\;\;0 \leq x \leq \pi$\\
		
		29. Express the area under the curve $y=x^5$ from 0 to 2 as a limit, then evaluate the limit.\\
		
		\textbf{5.2}\\
		
		Express the limit as a definite integral on the given interval.\\
		17. $\displaystyle \lim_{n \to \infty} \sum_{i=1}^n \frac{3^{x_i}}{1+x_i} \Delta x,\;\;\; \left[ 0, 1\right]$\\
		19. $\displaystyle \lim_{n \to \infty} \sum_{i=1}^n \left[ 5(x_i^*)^3 - 4x_i^*\right] \Delta x,\;\; \left[ 2, 7\right]$\\
		
		Express the integral as a limit of Reimann sums. Do not evaluate the limit.\\
		29. $\int_1^3 \sqrt{4 + x^2} \; dx$\\
		
		Evaluate the integral.\\
		\begin{minipage}[t]{0.5\textwidth}
			21. $\int_2^5 (4-2x) dx$
		\end{minipage}
		\begin{minipage}[t]{0.5\textwidth}
			25. $\int_0^1 (x^3 - 3x^2) dx$\\
		\end{minipage}
		
		\textbf{5.3}\\
		
		Use FTC1 to find the derivative of the function.\\
		7. $g(x) = \int_0^x \sqrt{t + t^3} \; dt$\\
		11. $f(x) = \int_x^0 \sqrt{1 + \sec t} \; dt$\\
		
		Evaluate the integral.\\
		\begin{minipage}[t]{0.3\textwidth}
			23. $\int_1^9 \sqrt x \; dx$\\
			38. $\int_0^1 \cosh t \; dt$\\
		\end{minipage}
		\begin{minipage}[t]{0.3\textwidth}
			26. $\int_{-5}^5 e\; dx$\\
			38. $\int_0^4 2^s \; ds$
		\end{minipage}
		\begin{minipage}[t]{0.3\textwidth}
			33. $\int_0^1 (1+4)^3 \; dr$
		\end{minipage}\\
		
		Find the derivative of the function.\\
		\begin{minipage}[t]{0.5\textwidth}
			61. $F(x) = \int_x^{x^3} e^{t^2}\; dt$
		\end{minipage}
		\begin{minipage}[t]{0.5\textwidth}
			61. $Y = \int_{\cos x}^{\sin x} \ln(1+2v) \; dv$
		\end{minipage}
	\end{itemize}
	\clearpage
	\large\underline{\textbf Assignment 11:}
	\begin{itemize}
		\item[] \textbf{5.4}\\
		
		Find the general indefinite integral.\\
		\begin{minipage}[t]{0.5\textwidth}
			8. $\int (u^6 - 2u^5 - u^3 + \frac27)\; du$\\
			16. $\int \sec t(\sec t + \tan t)\; dt$\\
		\end{minipage}
		\begin{minipage}[t]{0.5\textwidth}
			13. $\int (\sin x + \sinh x)\; dx$
		\end{minipage}
		
		Evaluate the integral.\\
		\begin{minipage}[t]{0.5\textwidth}
			23. $\int_{-2}^{0} (\frac12 t^4 + \frac14t^3 - t) \; dt$\\
			30. $\int_0^1 \frac{4}{1+p^2}\; dp$
		\end{minipage}
		\begin{minipage}[t]{0.5\textwidth}
			27. $\int_0^\pi (5e^x + 3 \sin x) \; dx$\\
			36. $\int_0^\frac{\pi}{4} \sec \theta \tan \theta \; d \theta$\\
		\end{minipage}
		
		\textbf{5.5}\\
		
		Evaluate the indefinite integral.\\
		\begin{minipage}[t]{0.5\textwidth}
			13. $\int \frac{dx}{5-3x}$\\
			34. $\int \frac{\cos \frac{\pi}{x}}{
				x^2} \; dx$
		\end{minipage}
		\begin{minipage}[t]{0.5\textwidth}
			33. $\int (\cos (1+5t)) \; dt$\\
			45. $\int \frac{1+x}{1+x^2}\; dx$\\
		\end{minipage}
		
		Evaluate the definite integral.\\
		\begin{minipage}[t]{0.5\textwidth}
			54. $\int_0^1 (3t-1)^{50}$\\
			67. $\int_1^2 x\sqrt{x-1}\; dx$
		\end{minipage}
		\begin{minipage}[t]{0.5\textwidth}
			63. $\int_0^{13} \frac{dx}{\sqrt[3]{(1+2x)^2}}$\\
			73. $\int_0^1 \frac{dx}{(1+\sqrt x)^4}$\\
		\end{minipage}\\
		
		\textbf{6.1}\\
		
		Find the area enclosed by the given curves.\\
		\begin{minipage}[t]{0.3\textwidth}
		14. $y = x^2$\\
		19. $y = \cos(\pi x)$\\
		25. $y = x^4$\\
		31. $y = \frac{x}{1+x^2}$
		\end{minipage}
		\begin{minipage}[t]{0.6\textwidth}
			$y = 4x - x^2$\\
			$y = 4x^2 - 1$\\
			$y = 2 - |x|$\\
			$y = \frac{x^2}{1+x^3}$
		\end{minipage}\\
	\end{itemize}
\end{document}