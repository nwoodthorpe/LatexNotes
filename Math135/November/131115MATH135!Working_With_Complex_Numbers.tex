\documentclass{letter}
\usepackage[margin=0.75in]{geometry}
\usepackage{amsmath}
\usepackage{amssymb}
\usepackage{enumerate}
\usepackage{changepage}

\begin{document}
	\begin{center}
		\LARGE Math135 - November 13, 2015\\
		\large Working With Complex Numbers
	\end{center}
	\vspace{0.25 in}
	\underline{\textbf{Properties of Arithmetic in $\mathbb{C}$}}
	
	Let $u, v, w \in \mathbb{C}$
	\begin{enumerate}[1)]
		\item Associativity of addition: $(u + v) + z = u (v z)$
		\item Commutativity of addition: $u + v = v + u$
		\item Additive identity: $0 = 0 + 0i$ has the property that $z + 0 = z$
		\item Additive inverse: If $z = x+yi$ then there exists an \textit{additive inverse} of $z$, written $-z$ with the property that $z + (-z) = 0$ The additive inverse of $z = x+yi$ is $-z = -x - yi$
		\item Associativity of multiplication: $(u \cdot v) \cdot z = u \cdot (v \cdot z)$
		\item Commutativity of multiplication: $u \cdot v = v \cdot u$
		\item Multiplicative identity: $1 = 1 + 0i$ has the property that $z \cdot 1 = z$
		\item Multiplication inverses: If $z = x + yi \neq 0$ then there exists a \textit{multiplicative inverse} of $z$, written $z^{-1}$, with the property  that $z \cdot z^{-1} = 1$. The multiplicative inverse of $z = x + yi$ is $z^{-1} = \frac{x-yi}{x^2 + y^2}$
		\item Distributivity: $z \cdot (u + v) = z \cdot u + z \cdot v$
	\end{enumerate}
	
	Prove Number 2: $u + v = v + u$\\
	
	Let $u = x = yi, x, y \in \mathbb{R}$\\
	Let $v = a + bi, a, b \in \mathbb{R}$
	
	Left Side:\\
	$= x + yi + a + bi$\\
	$= (x+a) + (y+b)i$\\
	$= (a+x) + (b+y)i$\\
	$= a + bi + x + yi$\\
	$= v + u$\\
	$=$ RS
	
	\underline{\textbf{Exponents}}
	
	By definition: $z^0 = 1 + 0i$ and $z^n = z \cdot z \cdot \dots \cdot z$ n times.\\
	Given these definitions, we can define the following: (If $m, n \in \mathbb{N}$)
	
	$z^m \cdot z^n = z^{m+n}$\\
	${z^m}^n = z^{mn}$
	
	Example: Find a real solution to $6z^3 + (1+3\sqrt{2}i)z^2 - (11-2\sqrt{2}i)z - 6 = 0$
	\begin{flalign*}
		6z^3 + z^2 + 3\sqrt{2}z^2i - 11z + 2\sqrt{2}zi - 6 &= 0&\\
		6x^3 + z^2 - 11z - 6 + (3\sqrt{2}z^2 + 2\sqrt{2}z)i &= 0
	\end{flalign*}
	\begin{flalign*}
		3\sqrt{2}z^2 + 2\sqrt{2}z &= 0&\\
		\sqrt{2}z(3z + 2) &= 0\\
		z &= 0, \frac{-3}{2}
	\end{flalign*}
	Now we'll sub these values into the other function.
	\begin{flalign*}
		6(0)^3 + 0^2 -11(0) - 6 &= 0&\\
		-6 &= 0\\
		6(\frac{-2}{3})^3 + (\frac{-2}{3})^2 + 11\frac{-2}{3}- 6 &= 0\\
		0 &= 0\\
	\end{flalign*}
	$\therefore$ Our solution is $z=6$
	
	\underline{\textbf{Complex Conjugate}}
	
	The complex conjugate of $z = x + yi$ is the complex number $x - yi$, denoted by $\bar{z}$.
	
	Example: $z = 1+2i, \bar{z} = 1 - 2i$
	
	We can express $z^{-1}$ in terms of $\bar{z}$
	
	If $z = x+yi$
	
	$z^{-1} = \dfrac{x - yi}{x^2 + y^2}$\\
	$z^{-1} = \dfrac{x-yi}{(x-yi)(x+yi)}$\\
	$z^{-1} = \dfrac{\bar{z}}{z\bar{z}}$
	
	\underline{\textbf{Properties of Conjugates}}
	
	\begin{enumerate}[1)]
		\item $\overline{z + w} = \bar{z} + \bar{w}$
		\item $\overline{zw} = \bar{z} \cdot \bar{w}$
		\item $\bar{\bar{z}} = z$
		\item $z + \overline{z} = 2Re(z)$
		\item $z - \overline{z} = 2iIe(z)$
		\item $\overline{(\dfrac{1}{z})}= \dfrac{1}{z}$
		\item $\overline{(\dfrac{z}{w})} = \dfrac{\bar{z}}{\bar{w}}$
	\end{enumerate}
	
	\underline{\textbf{Real Numbers and Conjugates}}
	
	Prove that $z \in \mathbb{R}$ iff $z = \overline{z}$.
	\begin{itemize}
		\item[$\implies$] Assume $z$ is a real value:
		
		$z = x + 0i$\\
		$\overline{z} = x - 0i$\\
		$z = \overline{z}$
		\item[$\impliedby$] Assume $z = \overline{z}$:
		
		$x + yi = x-yi$\\
		$y = -y$\\
		$y = 0$\\
		So, $z = x + 0i$, $z \in \mathbb{R}$.
	\end{itemize}
	
	Note: $z$ is purely imaginary iff $z = -\overline{z}$
\end{document}