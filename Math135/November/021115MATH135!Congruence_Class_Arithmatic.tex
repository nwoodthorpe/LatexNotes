\documentclass{letter}
\usepackage[margin=0.75in]{geometry}
\usepackage{amsmath}
\usepackage{amssymb}
\usepackage{enumerate}

\begin{document}
	\begin{center}
		\LARGE Math135 - November 2'nd, 2015\\
		\large Integers Modulo M - Congruence Class Arithmetic
	\end{center}
	\vspace{0.25 in}
	\underline{\textbf{Recall:}}
	
	$ax \equiv c \;(\bmod\; m)$ has a solution if $\gcd(a, m) \vert c$.\\
		
	\underline{\textbf{Congruence Classes}}
	
	Let $a, m \in \mathbb{Z}$ where $m > 0$. The  \textbf{Congruence Class Modulo $m$} of the integer $a$ is the set of integers\\
	$[a] = {x \in \mathbb{Z} \vert x \equiv a}$\\
	
	Example: Suppose $m = 5$
	\begin{enumerate}[ ]
		\item $[0] = \lbrace\dots -10, -5, 0, 5, 10, \dots\rbrace = \lbrace5k : k \in \mathbb{Z}\rbrace$
		\item $[1] = \lbrace\dots, -9, -4, 1, 6, 11, \dots\rbrace = \lbrace 5k+1 : k \in \mathbb{Z}\rbrace$
		\item And so on.
	\end{enumerate}
	\underline{$\mathbf{\mathbb{Z}_m}$}
	
	Let $m $ be a positive integer. We define $\mathbb{Z}_m$ to be the set of $m$ congruence classes.\\

	$\mathbb{Z}_m = \lbrace[0], [1], [2], \dots, [m+1]\rbrace$\\

	\underline{\textbf{Congruence Class Arithmetic}}
	
	We will define addition and multiplication of Congruence Classes as follows:
	\begin{enumerate} [ ]
		\item $[a] + [b] = [a + b]$
		\item $[a][b] = [ab]$
	\end{enumerate}
	
	Examples in $\mathbb{Z}_6$:
	\begin{enumerate} [i)]
		\item $[14] = [2]$
		\item $[13] = [5]$
		\item $[2][5] = [10] = [4]$
		\item $[14][-13] = [-182] = [-2] = [4]$
	\end{enumerate}
	Addition and multiplication are well defined. That is, in any congruence class addition or multiplication, a congruence class can be replaced with an equivalent congruence class without disrupting the equality.
	
	\underline{\textbf{Identities}}
	
	Given a set and an operation, an identity is "something that does nothing". More formally, given a set $S$, and an operation $\circ$, an identity is an element $e \in S$ such that $\forall a \in S, a \circ e = a$.
	
	The additive identity is $[0]$ as $[a] + [0] = [a + 0] = [a]$.\\
	The multiplicative identity is $[1]$ as $[a][1] = [1a] = [a]$.\\
	\pagebreak
	
	\underline{\textbf{Inverse}}
	
	The element $b \in S$ is an inverse of $a \in S$ if $a \circ b = b \circ a = e$ where e is the corresponding identity for the operation $\circ$.
	
	For example, the \textbf{additive} identity of $[5]$ is $[-5]$ as $[5] + [-5] = [0] = e$ (The additive identity).
	
	We define subtraction as addition by the inverse. We will always be able to find an additive inverse.\\
	We define division as multiplication by the inverse. We will \textbf{NOT} always be able to find the multiplicative inverse. We denote the multiplicative inverse of some value $[a]$ as $[a]^{-1}$.
	
	Examples in $\mathbb{Z}_4$:
	\begin{enumerate} [i)]
		\item $[3]^{-1} = [3]$ since $[3][3] = [9] = [1]$
		\item $[2]^{-1}$ does not exist since $[2][x] = 1$ has no solution in $\mathbb{Z}_4$
	\end{enumerate}
	
	\underline{\textbf{Practice:}}
	
	Solve the following in $\mathbb{Z}_1{14}$
	
	\begin{enumerate} [i)]
		\item 
		\begin{flalign*}
			[75] - [x] &= [5]&\\
			[75] + [-x] &= [50] \text{ \textbf{(We must express subtraction as addition)}}\\
			[75] &= [50] + [x] \text{ \textbf{(By additive identity)}}\\
			[75] + [-50] &= [x] \text{\textbf{ (By additive identity)}}\\
			[25] &= [x]\\
			[11] &= [x] \text{ \textbf{(Always reduce to lowest terms)}}
		\end{flalign*}
		\item 
		\begin{flalign*}
			[10][x] &= [1]&\\
			10x &= 1\;(\bmod\;14)\\
			10x + 14y =& 1
			\end{flalign*}
			\text{Notice how this is a simple Linear Diophantine. We know how to solve for x and y.}\\
			$\gcd(10, 14) = 2$\\
			$2 \nmid 1 \therefore$ \text{There is no solution}
		\item 
		\begin{flalign*}
			[10][x] &= 2&\\
			10x &= 2\;(\bmod\; 14)\\
			10x + 14y &= 2\\
			\gcd(10, 14) &= 2\\
		\end{flalign*}
		$2\vert2$ so there is a solution.\\
		$x = -4, y = 3$ (By inspection)
		Just like when solving a Linear Diophantine, we now need to plug this into the general formula to get all values of x.\\
		\begin{flalign*}
			x &= -4 + 7n, n\in\mathbb{Z} \text{ (By LDET2)}&\\
			x &\equiv -4 \;(\bmod\; 7)\\
			x &\equiv 3 \;(\bmod\; 7)\\
			x &\equiv 3, 10 \;(\bmod\; 7)
		\end{flalign*}
		$\therefore [x] = [3], [10]$
	\end{enumerate}
\end{document}