\documentclass{letter}
\usepackage[margin=0.75in]{geometry}
\usepackage{amsmath}
\usepackage{amssymb}
\usepackage{enumerate}
\usepackage{changepage}

\begin{document}
	\begin{center}
		\LARGE Math135 - November 25, 2015\\
		\large More Polynomials: RT and FT
	\end{center}
	\vspace{0.25 in}
	\underline{\textbf{Example:}}
	
	\begin{itemize}
		\item[\;\;] Let $f(x)$ and $g(x)$ be polynomials in a field $\mathbb{F}$.\\\\
		If $f(x) \;\vert\; g(x)$ and $g(x)\;\vert\; f(x)$, show that $f(x) = c g(x)$ for some $c$ in the field $\mathbb{F}$.\\
		\begin{itemize}
			\item[Proof: ] Assume $f(x) \mid g(x)$ and $g(x) \mid f(x)$\\
			$g(x) = _1(x)f(x)$ for some $g_q(x) \in \mathbb{F}\left[x\right]$\\
			$f(x) = q_2g(x)$ for some $q_2(x) \in \mathbb{F}\left[x\right]$\\
			
			By substitution, $g(x) = q_1(x)q_2(x)g(x)$\\
			We know the degree of $q_1(x)q_2(x) = 0$ (Since $(q_1(x)q_2(x) = 1$)\\
			So, the degree of $q_1(x) = 0$, and the degree of $q_2(x) = 0$\\
			This means $q_1(x)$ and $q_2(x)$ are constants.\\
			$q_2(x) = c,$ for some $c \in \mathbb{F}$.\\\\
			$\therefore f(x) = c g(x)$
		\end{itemize}
	\end{itemize}
	\underline{\textbf{Polynomial Equation}}
	
	A polynomial equation is an equation of the form:
	\[ a_nx^n + a_{n-1}x^{n-1}+ \dots + a_1x + a_0 = 0 \]
	Which will often be written as $f(x) = 0$/
	
	An element $c \in \mathbb{F}$ is called a root or zero of $f(x)$ iff $f(c) = 0$.
	
	\underline{\textbf{Remainder Theorem}}
	
	The remainder when the polynomial $f(x)$ is divided by $(x-c)$ is $f(c)$.
	
	\begin{itemize}
		\item[Ex. ]
		\begin{enumerate}[i)]
			\item From yesterday:\\\\
			$f(z) = iz^3 + (3+i)z^2 + (3+5i)z + (-2+2i)$\\
			$f(z) = (iz^2 + 4z + (-1-i))(z + (1+i)) + 2i$\\\\
			RT tells us $f(-1-i) = 2i$
			
			\item $f(x) = x^2 + 1 = (x+1)(x-1) +2$\\\\
			RT tells us $f(-1) = 2$, $f(1) = 2$.
		\end{enumerate}
		\item[Ex. ] In $\mathbb{Z}_7$, what is the remainder when $f(x) = 4x^3 + 2x + 5$ is divided by $x + 6$\\\\
		\begin{minipage}[t]{0.5\textwidth}
			The dumb way:
			\begin{flalign*}
			f(x) &= f(-6)&\\
			&= 4(-6)^3 + 2(-6) + 5\\
			&= -864 + 2 + 5\\
			&= 4
			\end{flalign*}
		\end{minipage}
		\begin{minipage}[t]{0.5\textwidth}
			The smart way:
			\begin{flalign*}
				f(-6) &= f(1)&\\
				&= 4 + 2 + 5\\
				&= 4
			\end{flalign*}
		\end{minipage}
	\end{itemize}
	\underline{\textbf{Factor Theorem (FT)}}
	
	The linear polynomial $(x-c)$ is a factor of the polynomial $f(x)$ iff $f(c ) = 0$.
\end{document}