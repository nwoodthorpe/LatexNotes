\documentclass{letter}
\usepackage[margin=0.75in]{geometry}
\usepackage{amsmath}
\usepackage{amssymb}
\usepackage{enumerate}
\usepackage{changepage}
\usepackage{tikz}
\usepackage{pgfplots}
\pgfplotsset{compat=1.8}

\pgfplotsset{vasymptote/.style={
		before end axis/.append code={
			\draw[densely dashed] ({rel axis cs:0,0} -| {axis cs:#1,0})
			-- ({rel axis cs:0,1} -| {axis cs:#1,0});
		}
	}}
	
\newcommand{\m}{\begin{bmatrix}}
\newcommand{\mm}{\end{bmatrix}}
\newcommand{\0}[1]{\begin{bmatrix}#1\end{bmatrix}}
\newcommand{\h}[1]{\underline{\textbf{#1}}}	

\begin{document}
	\begin{center}
		\LARGE Math136 - January 22'th, 2016\\
		\large Solving Systems of Equations
	\end{center}
	\vspace{0.25 in}
	
	\h{Equivalence}

	Two systems of linear equations which have the same solution sets are \textbf{equivalent}.
	
	\h{(Augmented) Coefficient Matrix}
	
	Suppose we have a system of $m$ equations in $n$ variables:
	
	$a_{11}x_1 + a_{12}x_2 + \dots + a_{1n}x_n = b_1$\\
	$a_{21}x_1 + a_{22}x_2 + \dots + a_{2n}x_n = b_2$\\
	$\vdots$\\
	$a_{m1}x_1 + a_{m2}x_n + \dots + a_{mn}x_n = b_n$
	
	The coefficient matrix of this system is:
	
	$\0{a_{11}&a_{12}&\dots&a_{1n}\\a_{21}&a_{22}&\dots&a_{2n}\\\vdots\\a_{m1}&a_{m2}&\dots&a_{mn}}$
	
	And the augmented coefficient matrix is:
	
	$\left[\begin{array}{cccc|c}
		a_{11}&a_{12}&\dots&a_{1n}&b_1\\a_{21}&a_{22}&\dots&a_{2n}&b_2\\\vdots&&&&\vdots\\a_{m1}&a_{m2}&\dots&a_{mn}&b_m
	\end{array}\right]$
	
	For now, we can think of the augmented matrix as a convenient way to represent our system of equations. Later, we'll see that we get a lot more out of this by developing matrix operations.
	
	\h{Elementary Row Operations}
	
	The elementary row operations are:
	
	\begin{enumerate}
		\item Multiplying a row by a non-zero scalar
		\item Adding a multiple of one row to another
		\item Swapping two rows
	\end{enumerate}
	
	Note: All of these operations are reversible.
	
	As we will see, if we start with a system of equations, then perform elementary row operations on the augmented matrix, we end up with the augmented matrix of an equivalent system of equations.
	
	\h{Row Equivalent}
	
	Two matrices $A$ and $B$ are row-equivalent if one can be obtained from the other by a sequence of elementary row operations.
	
	\h{Theorem 2.2.1}
	
	If the augmented matrices [$ A_1 \mid \vec b_1 $] and [$A \mid \vec b$] are row equivalent, then the associated systems of linear equations are equivalent.
	\clearpage
	\h{Reduced Row Echelon Form}
	
	A matrix $R$ is in reduced row echelon form (RREF) if:
	
	\begin{enumerate}
		\item All rows containing a non-zero entry are above rows which contain only zero's.
		\item The first non-zero entry in each row is 1 (The leading 1)
		\item The leading in each non-zero row is to the right of the leading row in any row above
		\item A leading one is the only non-zero entry in it's column.
	\end{enumerate}
	
	If $A$ is row equivalent to $R$ which is in RREF, then $R$ is the RREF of $A$.
\end{document}