\documentclass{letter}
\usepackage[margin=0.75in]{geometry}
\usepackage{amsmath}
\usepackage{amssymb}
\usepackage{enumerate}
\usepackage{changepage}
\usepackage{tikz}
\usepackage{pgfplots}
\pgfplotsset{compat=1.8}

\pgfplotsset{vasymptote/.style={
		before end axis/.append code={
			\draw[densely dashed] ({rel axis cs:0,0} -| {axis cs:#1,0})
			-- ({rel axis cs:0,1} -| {axis cs:#1,0});
		}
	}}
	
\newcommand{\m}{\begin{bmatrix}}
\newcommand{\mm}{\end{bmatrix}}
\newcommand{\0}[1]{\begin{bmatrix}#1\end{bmatrix}}
\newcommand{\h}[1]{\underline{\textbf{#1}}}	

\begin{document}
	\begin{center}
		\LARGE Math136 - January 8'th, 2016\\
		\large Linear Dependency and Basis
	\end{center}
	\vspace{0.25 in}
	
	\underline{\textbf{Proof Continued From Last Class}}
	
	\begin{itemize}
		\item[($\implies$)] We can express $\vec v_i$ as\\
		\begin{flalign*}
			\vec v_i &= 0\vec v_1 + \dots + 1\vec v_i + 0\vec v_{i+1} + \dots + \vec v_k \in \text{ span}(\{ \vec v_1, \dots, \vec v_k \})&\\
			&= \text{span}(\{ \vec v_1, \dots, \vec v_{i-1}, \vec v_{i+1}, \dots, \vec v_k \})
		\end{flalign*}
		And so $\vec v$ is a linear combination of $\vec v_1, \dots, \vec v_{i-1}, \vec v_{i+1}, \dots, \vec v_K$ by the definition of span.
		\item[E.g.] Try to reduce span$(\{ \begin{bmatrix}1\\0\end{bmatrix}, \begin{bmatrix}2\\0\end{bmatrix}, \begin{bmatrix}1\\3\end{bmatrix}, \begin{bmatrix}0\\1\end{bmatrix} \})$
		\begin{flalign*}
			&= \text{ span}(\{ \begin{bmatrix}1\\0\end{bmatrix}, \begin{bmatrix}1\\3\end{bmatrix}, \begin{bmatrix}0\\1\end{bmatrix} \})\;\;\;\;\;\text{since $\begin{bmatrix}2\\0\end{bmatrix} = 2\begin{bmatrix}1\\0\end{bmatrix} + 0\begin{bmatrix}1\\3\end{bmatrix}+ 0\begin{bmatrix}0\\1\end{bmatrix}$}&\\
			&= \text{ span}(\{ \begin{bmatrix}1\\0\end{bmatrix}, \begin{bmatrix}0\\1\end{bmatrix} \})\;\;\;\;\;\;\;\;\;\;\;\;\;\text{since $\begin{bmatrix}1\\3\end{bmatrix} = 1\begin{bmatrix}1\\0\end{bmatrix} + 3\begin{bmatrix}0\\1\end{bmatrix}$}
		\end{flalign*}
		
		Of course, there is no way to create either vector from a scalar multiple of the other in this case, so we're done.
	\end{itemize}
	
	\underline{\textbf{Linear Dependency}}
	
	A set of vectors $\{ \vec v_1, \dots, \vec v_k \}$ is \textbf{linearly dependent} if there are scalars $c_1, \dots, c_k \in \mathbb{R}$ which are not all 0 such that $c_1\vec v_1 + \dots + c_k \vec v_k = 0$
	
	\begin{itemize}
		\item[E.g. ] (Referring to previous e.g.) we have:
		\begin{flalign*}
			&2\m1\\0\mm + (-1)\m 2\\0 \mm + 0 \m 1\\3 \mm + 0 \m 0\\1\mm = \m 0\\0\mm&\\
			&\therefore \{ \m 1\\0\mm, \m 2\\0\mm, \m 1\\3\mm, \m0\\1\mm \} \text{ is linearly dependent.}
		\end{flalign*}
		Note that this linear dependency allowed us to show that $\m2\\0\mm$ could be removed.
	\end{itemize}
	
	\h{Linear Independency}
	
	If $\{ \vec v_1, \dots, \vec x_k \}$ is not linearly dependent, then it is \textbf{linearly independent}.
	
	\begin{itemize}
		\item[E.g. ] The set of $\{ \m 1\\0\mm, \m0\\1\mm \}$ is linearly independent. To check:\\\\
		Suppose that $c_1\m1\\0\mm + c_2\m0\\1\mm = \m0\\0\mm$\\
		$\m c_1\\c_2 \mm = \m0\\0\mm$, so $c_1 = 0, c_2 = 0$\\\\
		Hence, only the trivial linear combination (The solution where everything is 0) of these vectors gives $\vec 0$\\\\
		Note: Any set containing $\vec 0$ is linearly dependent.
	\end{itemize}
	
	\clearpage
	
	\h{Spanning Set and Basis}
	
	If $S \subseteq \mathbb{R}^n$ satisfies $S =$ span$(\{ \vec v_1, \dots, \vec v_k \}$ then $\{ \vec v_1, \dots, \vec v_k \}$ is called a \textbf{spanning set} for $S$. Also, $S$ is spanned by $\{ \vec v_1, \dots, \vec v_k \}$. If $\{ \vec v_1, \dots, \vec v_k \}$ is also linearly independent, then it is a \textbf{basis} for $S$.
	
	Note: if the spanning set is not linearly independent, we can use Theorem 2 to reduce to a smaller spanning set, and eventually a basis.
	
	\h{Standard Basis}
	
	The \textbf{standard basis} for $\mathbb{R}^n$ consists of $\vec c_1, \dots, \vec e^n$ where $\vec v_i \in \mathbb{R}^n$ has it's $i$'th coordinate equal to 1 and all other coordinates equal to 0.
	
	E.g. $\{ \0{1\\0\\0}, \0{0\\1\\0}, \0{0\\0\\1} \}$ is the standard basis for $\mathbb{R}^3$
	\begin{itemize}
		\item[E.g.] Find a basis for $S =$ span$(\{ \0{1\\1\\1}, \0{2\\1\\-1}, \0{0\\-1\\-3} \})$\\\\
		Let's check for linear dependency.\\\\
		$c_1\0{1\\1\\1} + c_2\0{2\\1\\-1} + c_3\0{0\\-1\\3} = \0{0\\0\\0}$\\
		$\implies \0{c_1 + 2c_2 + 0c_3\\c_1+c_2-c_3\\c_1-c_2-3c_3} = \0{0\\0\\0}$\\\\
		$c_1 + 2c_2 = 0\;\;\;\;\;\;$ (Equation 1)\\
		$c_1 + c_2 - c_3 = 0$ (Equation 2)\\
		$c_1 - c_2 - 3_3 = 0$ (Equation 3)\\\\
		Let's solve this with some high-school math.\\\\
		From eq. 1, $c_1 = -2c_2$\\
		Substitute into eq. 2\\
		$-2 c_2 + c_2 - c_3 = 0 \implies c_3 -c_2$\\
		Substitute both into eq. 3\\
		$(-2 c_2) - c_2 - 3(-c_2) = 0$\\
		So, in fact the solutions are given by $\0{c_1\\c_2\\c_2} = \0{-2t\\t\\-t}$ for any $t \in \mathbb{R}$\\\\
		Suppose we looked at $t=1$. We would see that:\\
		$-2\0{1\\1\\1} + 1\0{2\\1\\-1} + (-1)\0{0\\-1\\-3} = \0{0\\0\\0}$\\\\
		So we rearrange to get:\\
		$2\0{1\\1\\1} = \0{2\\1\\-1} + \0{1\\-1\\-3}$\\
		$\0{1\\1\\1} = \frac12 \0{2\\1\\-1} + \frac12 \0{0\\-1\\-3}\;\;\;\;$\\\\
		So by theorem 2, we can remove $\0{1\\1\\1}$ and we find $S$ = span$\{ \0{2\\1\\-1}, \0{0\\-1\\-3} \}$ which is a basis for $S$.
		\item[E.g. ] Describe the set $\0{1\\1} + \tau\0{1\\-1} \mid \tau \in \mathbb{R}$ geometrically.\\\\
		This is a line with slope (-1) drawn through $\0{1\\1}$ as shown below.\\\\
		
		\begin{center}
			\begin{tikzpicture}
			\begin{axis}[
			axis equal image,
			axis lines=middle,
			xmin=-3,xmax=3,
			ymin=-3,ymax=3,
			enlargelimits={abs=1cm},
			axis line style={latex-latex},
			yticklabel style={anchor=west},
			ytick={2},
			xtick={2},
			]
			% This doesn't clip to y=-10:10 nicely
			% because there are too few samples near the asymptote:
			\addplot[very thick, black, domain=-10:10,samples=200, restrict y to domain=-100:100]
			{(2 + (-1) * x};
			\addplot[only marks] table {
				1 1
			};
			\end{axis}
			\end{tikzpicture}
		\end{center}
	\end{itemize}
	
	\h{Line in $\mathbb{R}^n$}
			
	A \textbf{line} in $\mathbb{R}^n$ is given by the vector equation:
	
	$\vec x = c_1 \vec v + \vec b$
	
	For $c_1 \in \mathbb{R}, \vec v \in \mathbb{R}^n, \vec b \in \mathbb{R}^n$ and fixed.
\end{document}