\documentclass{letter}
\usepackage[margin=0.75in]{geometry}
\usepackage{amsmath}
\usepackage{amssymb}
\usepackage{enumerate}
\usepackage{changepage}
\usepackage{tikz}
\usepackage{pgfplots}
\pgfplotsset{compat=1.8}

\pgfplotsset{vasymptote/.style={
		before end axis/.append code={
			\draw[densely dashed] ({rel axis cs:0,0} -| {axis cs:#1,0})
			-- ({rel axis cs:0,1} -| {axis cs:#1,0});
		}
	}}
	
\newcommand{\m}{\begin{bmatrix}}
\newcommand{\mm}{\end{bmatrix}}
\newcommand{\0}[1]{\begin{bmatrix}#1\end{bmatrix}}
\newcommand{\h}[1]{\underline{\textbf{#1}}}	

\begin{document}
	\begin{center}
		\LARGE Math136 - January 13'th, 2016\\
		\large Dot Product and Orthogonality
	\end{center}
	\vspace{0.25 in}
	
	\h{Fact:}
	
	A K-Flat in $\mathbb{R}^n$ that passes through the origin is a subspace.
	
	\begin{itemize}
		\item[E.g. ] Find a basis for the subspace $S = \{ \0{x_1\\x_2\\x_3} \in \mathbb{R} \mid x_1 + x_ + x_3 = 0 \}$\\\\
		Note: if $x_1 + x_2 + x_3 = 0$, then $x_3 = -x-1 - x_2$\\\\
		So,
		\begin{flalign*}
			S &= \{ \0{x_1\\x_2\\-1_1-x_2} \in \mathbb{R}^3 \mid x_1, x_2 \in \mathbb{R} \}&\\
			&= \{ \0{x_1\\0\\-x_1} + \0{0\\x_2\\-x_2}\mid x_1, x_2 \in \mathbb{R} \}\\
			&= \{ x_1\0{1\\0\\1} + x_2\0{0\\1\\1} \}\\
			&= \text{span}\{ \0{1\\0\\-1}, \0{0\\1\\-1} \}
		\end{flalign*}\\\\
		So if we find that this span is L.I. then it will be a basis. We will use the fact that a set of two vectors is L.I. iff neither vector is a scalar multiple of eachother. You should prove this, but we can see of course this is the case, so we have a basis.
		
	\end{itemize}
		\h{Dot Product}
		
		Recall in $\mathbb{R}^2, \mathbb{R}^3$ we have dot products:
		
		$\0{x_1\\x_2} \cdot \0{y_1\\y_2} = x_1y_1 + x_2y_2$\\
		
		$\0{x_1\\x_2\\x_3} \cdot \0{y_1\\y_2\\y_3} = x_1y_1 + x_2y_2 =+ x_3y_3$
		
		We can now generalize this to $\mathbb{R}^n$
		
		The \textbf{Dot Product} of $\vec x = \0{x_1\\\vdots\\x_n} \in \mathbb{R}^n$ and $\vec y = \0{y_1\\\vdots\\y_n} \in \mathbb{R}^n$ is:
		\begin{flalign*}
		&\vec x + \vec y = x_1y_1 + \dots + x_ny_n = \sum_{i=1}^n x_iy_i&
		\end{flalign*}
		
		Note: The dot product is also called the standard inner product or the scalar product.
		
		\begin{itemize}
			\item[E.g. ] $\0{1\\1\\1\\1} \cdot \0{2\\-1\\-3\\-4} = 1\cdot 2 + 1\cdot(-1) + 1 \cdot (-3) + 1 \cdot (-4) = -6$
		\end{itemize}
	
	Notice the dot product always gives a \textbf{scalar}.
	
	\h{Theorem 1.3.2}
	
	If $\vec x, \vec y, \vec z \in \mathbb{R}^n$ and $s, t \in \mathbb{R}$, then:
	\begin{enumerate}[1)]
		\item $\vec x \cdot \vec x \geq 0$ and $\vec x \cdot \vec x = \vec 0$ iff $\vec x = \vec 0$
		\item $\vec x \cdot \vec y = \vec y \cdot \vec x$
		\item $\vec x \cdot (s \vec x + t \vec z) = s(\vec x \cdot \vec y) + t(\vec x \cdot \vec z)$
	\end{enumerate}
	
	\h{Norm}
	
	The \textbf{length} or \textbf{norm} of $\vec x \in \mathbb{R}^n$ is $\mid \mid \vec x \mid \mid$ = $\sqrt{\vec x \cdot \vec x}$
	
	\h{Orthogonal}
	
	$\vec x, \vec y \in \mathbb{R}^n$ are \textbf{orthogonal} if $\vec x \cdot \vec y = 0$
	
	Note: $\vec 0$ is orthogonal to any $\vec x \in \mathbb{R}^n$
	
	\h{Orthogonal Set}
	
	A set of vectors $\{ \vec v_1, \dots, \vec v_k \}$ in $\mathbb{R}^n$ is an orthogonal set iff $\vec v_i \cdot \vec v_j = 0$ for all $i \neq j, i, j \in \{ 1, \dots, k\} $
	
	For example, the standard basis for $\mathbb{R}^n$ is an orthogonal set.
	
	\h{Unit Vector}
	
	A vector $\vec x \in \mathbb{R}^n$ with $\mid \mid \vec x \mid \mid = 1$ is called a unit vector.
	
	\h{Theorem 1.3.3}
	
	If $\vec x, \vec y \in \mathbb{R}^n$ and $c \in \mathbb{R}^n$, then:
	\begin{enumerate}[1)]
		\item $\mid \mid \vec x \mid \mid \geq 0,$ and $\mid\mid\vec x\mid\mid = 0$ iff$ \vec x = \vec 0$
		\item $\mid\mid c \vec x\mid\mid = \mid c \mid \;\;\mid\mid\vec x\mid\mid$
		\item $(\vec x \cdot \vec y)^2 \leq \mid\mid\vec x\mid\mid^2 \;\;\mid\mid\vec y\mid\mid^2$
		\item $\mid\mid \vec x + \vec y \mid\mid \leq \mid\mid \vec x \mid\mid + \mid\mid \vec y \mid\mid$
	\end{enumerate}
\end{document}